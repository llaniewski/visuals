\documentclass[14pt]{extarticle}
\usepackage[english]{babel}
\usepackage[utf8]{inputenc}
\usepackage[a4paper,margin=20mm]{geometry}
\usepackage{graphicx}
\usepackage{ccicons}
\usepackage{tikz}
\usepackage{fancyhdr}

\pagestyle{fancy}
\fancyhf{}
\rfoot{{\footnotesize Łukasz Łaniewski-Wołłk}~\ccbysa}

\renewcommand{\headrulewidth}{0pt}

\title{}
\author{Łukasz Łaniewski-Wołłk}

\setlength\parindent{0pt}

\newcommand{\img}[1]{\begin{tikzpicture}[remember picture,overlay]
\node[inner sep=0] at (current page.center)
{\includegraphics[width=\paperwidth]{#1}};
\end{tikzpicture}%
\vspace{21cm}}

\begin{document}
\img{img_i_hip.png}
\section*{Hyperbolic}
This is a {\bf Poincaré disk}. It is just one representation of a hyperbolic space.
Beautiful images of Poincaré disks and other mathematical ideas can be found in works of a Dutch artist {\bf M. C. Escher}.
\end{document}
